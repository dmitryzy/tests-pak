\Qvery 
Рассчитать тепловой эффект химической реакции при стандартных условиях. $$3Fe_{2}O_{3} (k)+CO (g)=2Fe_{3}O_{4} (k)+CO2 (g)$$
\endQvery
\Qvery 
Рассчитать тепловой эффект химической реакции при стандартных условиях. $$Fe_{3}O_{4} (k)+CO (g)=3FeO (k)+CO_{2} (g)$$
\endQvery
\Qvery 
Рассчитать тепловой эффект химической реакции при стандартных условиях. $$CH_{4}(g) + СO_{2}(g) = 2H_{2} (g)+2CO(g)$$
\endQvery
\Qvery 
Записать термохимическое уравнение реакции горения одного моля пропана $C_{3}H_{8}$(g), в результате которой образуются пары воды и диоксид углерода. Сколько теплоты выделится при сгорании 1 м$^{3}$ пропана в пересчете на нормальные условия?
\endQvery
