\Qvery 
Какое покрытие железа, оловянное или цинковое устойчиво к атмосферной коррозии при нарушении цельности покрытия? Выводы подтвердите цифровыми данными.
\endQvery
\Qvery 
Какое покрытие кобальта, оловянное или цинковое, устойчиво к атмосферной коррозии при нарушении цельности покрытия? Выводы подтвердите цифровыми данными.
\endQvery
\Qvery 
Как происходит атмосферная коррозия луженого железа и луженой меди при нарушении цельности покрытия? Составьте электронные уравнения анодного и катодного процессов.
\endQvery
\Qvery 
Приведите пример протекторной защиты железа в электролите, содержащем растворенный кислород. Составьте электронные уравнения анодного и катодного процессов. 
\endQvery
\Qvery 
Приведите пример протекторной защиты никеля в электролите, содержащем растворенный кислород. Составьте электронные уравнения анодного и катодного процессов.
\endQvery
\Qvery 
Приведите пример протекторной защиты цинка в электролите, содержащем растворенный кислород. Составьте электронные уравнения анодного и катодного процессов.
\endQvery
\Qvery 
Если гвоздь вбить во влажное дерево, то ржавчиной покрывается та его часть, которая находится внутри дерева. Чем это можно объяснить? Анодом или катодом является эта часть гвоздя? Составьте электронные уравнения соответствующих процессов. 
\endQvery